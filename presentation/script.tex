\documentclass[12pt]{report}
\newcommand\tab[1][1cm]{\hspace*{#1}}
% pre\'ambulo

\usepackage{lmodern}
\usepackage[T1]{fontenc}
\usepackage[spanish,activeacute]{babel}
\usepackage[margin=1in,left=0.7in,right=0.7in,top=1in]{geometry}
\usepackage{mathtools}
\usepackage{bussproofs}
\usepackage{lscape}
\usepackage{resizegather}
\usepackage{verbatim}
\usepackage[utf8]{inputenc}
\usepackage{amssymb}
\usepackage{hyperref}

\title{Tarea Ethereum}
\author{Jorge Solano}
\date{ }

\begin{document}
	\textbf{Definiciones}
	\begin{enumerate}
		\item \textbf{Test network: } \url{https://hyperledger-fabric.readthedocs.io/en/release-2.0/test_network.html}\\
		\item \textbf{Commercial paper: } \url{https://hyperledger-fabric.readthedocs.io/en/release-2.0/tutorial/commercial\_paper.html}\\
		\item \textbf{goleveldb:} LevelDB es una base de datos llave/valor en Go.\\
		\item \textbf{CouchDB:} Es una base de datos open source, orientada a documentos y NoSQL, implementada en Erlang.\\
		\item \textbf{logspout:} Es la bitácora de los routers de los contenedores que se ejecutan en Docker \url{https://github.com/gliderlabs/logspout#logspout}.\\
		\item \textbf{Commercial-paper:}: En el smart contract papercontract.js\\
		%\begin{figure}[h]
		%	\includegraphics[scale=.7]{vul_01}
		%	\centering
		%\end{figure}
	\end{enumerate}

\end{document}