\documentclass[12pt]{article}
\newcommand\tab[1][1cm]{\hspace*{#1}}
% pre\'ambulo

\usepackage{lmodern}
\usepackage[T1]{fontenc}
\usepackage[spanish,activeacute]{babel}
\usepackage[margin=1in,left=0.7in,right=0.7in,top=1in]{geometry}
\usepackage{mathtools}
\usepackage{bussproofs}
\usepackage{lscape}
\usepackage{resizegather}
\usepackage{verbatim}
\usepackage[utf8]{inputenc}
\usepackage{ amssymb }
\usepackage{hyperref}
\usepackage{titling}

\title{Introducción a Hyperledger\\Tarea}
\author{René Dávila - Jorge Solano}
\date{ }

\pretitle{%
	\begin{center}
		\LARGE
		\includegraphics[width=4cm,height=4cm]{iimas}\\[\bigskipamount]
	\end{center}
}
\posttitle{}

\begin{document}
	%\includegraphics [width =0.2 \textwidth ]{iimas}
	\maketitle
	
	\begin{enumerate}
		\item En Hyperledger Fabric siempre debe existir un servicio de ordenamiento para la administración de la red. Describe como funciona ordering service en Fabric (\url{https://hyperledger-fabric.readthedocs.io/en/release-2.0/orderer/ordering\_service.html}) (2.5 puntos).
		
		\item Hyperledger, por defecto, utiliza el algoritmo Raft para llevar a cabo el consenso en la red Fabric. Describe el algoritmo de consenso Raft (\url{https://raft.github.io}) (2.5 puntos).
		
		\item En la presentación papernet.pdf está el caso de uso Aplicación cliente (páginas 3-16). Tomando como base ese caso realiza lo siguiente (2.5 puntos):
		\begin{enumerate}
			\item Modifica el smart contract.\\
			En fabric-samples/chaincode/fabcar/javascript/fabcar.js agrega un método al smart contract (ya sea de consulta o de modificación de la blockchain). Muestra la función que agregaste.
			\item Invoca el método creado desde una aplicación cliente.\\
			Si es de consulta utiliza el archivo fabric-samples/fabcar/javascript/query.js, si es de modificación utiliza el archivo fabric-samples/fabcar/javascript/invoke.js. Muestra la invocación que realizaste al método y agrega una captura de pantalla con la salida obtenida.
		\end{enumerate}
	
		\item En la presentación papernet.pdf está el caso de uso Papel comercial (páginas 18-62). Tomando como base ese caso realiza lo siguiente (2.5 puntos):
		\begin{enumerate}
			\item Instala y aprueba el contrato de parte de MagnetoCorp.\\
			Sigue las instrucciones que vienen en la presentación de la página 18 a la página 61. Muestra los contenedores docker corriendo en tu máquina.
			\item Emite un papel comercial de parte de MagnetoCorp.\\
			Ahora, imita a Isabella y emite un papel comercial (páginas 46-51). Muestra qué pasa cuando invocas a issue.js y explica por qué pasa eso.
		\end{enumerate}
	\end{enumerate}

\end{document}