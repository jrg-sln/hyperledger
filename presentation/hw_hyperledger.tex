\documentclass[12pt]{article}
\newcommand\tab[1][1cm]{\hspace*{#1}}
% pre\'ambulo

\usepackage{lmodern}
\usepackage[T1]{fontenc}
\usepackage[spanish,activeacute]{babel}
\usepackage[margin=1in,left=0.7in,right=0.7in,top=1in]{geometry}
\usepackage{mathtools}
\usepackage{bussproofs}
\usepackage{lscape}
\usepackage{resizegather}
\usepackage{verbatim}
\usepackage[utf8]{inputenc}
\usepackage{ amssymb }
\usepackage{hyperref}
\usepackage{titling}

\title{Introducción a Hyperledger\\Tarea}
\author{René Dávila - Jorge Solano}
\date{ }

\pretitle{%
	\begin{center}
		\LARGE
		\includegraphics[width=4cm,height=4cm]{iimas}\\[\bigskipamount]
	\end{center}
\posttitle{

\begin{document}
	%\includegraphics [width =0.2 \textwidth ]{iimas}
	\maketitle
	
	\begin{enumerate}
		\item En Hyperledger Fabric siempre debe existir un servicio de ordenamiento para la administración de la red. Describe como funciona ordering service en Fabric (\url{https://hyperledger-fabric.readthedocs.io/en/release-2.0/orderer/ordering\_service.html}).
		
		\item Hyperledger, por defecto, utiliza el algoritmo Raft para llevar a cabo el consenso en la red Fabric. Describe el algoritmo de consenso Raft (\url{https://raft.github.io}).
	\end{enumerate}

\end{document}